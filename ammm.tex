\documentclass{article}
\usepackage{graphicx} % Required for inserting images
\usepackage{amsmath}
\usepackage{enumitem}
\usepackage[paper=a4paper, margin=2cm]{geometry}

\title{Algorithmic Methods for Mathematical Models\\
  Course Project }
\author{Tim Wichelmann\\ \texttt{tim.wichelmann@estudiantat.upc.edu}\\[1ex] % [1ex] adds vertical space
  Jakob Eberhardt\\ \texttt{jakob.eberhardt@estudiantat.upc.edu}}
\date{\today}

\begin{document}

\maketitle

\section{Introduction}

\subsection{Decision variables}
\begin{itemize}
    \item $y_{ij}$ is a matrix of binary variables which indicates if the order $i$ is being baked in time slot $j$.
\end{itemize}

\subsection{Auxiliary variables}

\begin{itemize}
\item $x_i$ is a binary variable indicating whether order $i$ has the right amount of time slots assigned to it. 
We use it as an indicator of 
\item $\mathit{start\_{ij}}$ denotes the time slot $j$ in which the baking process of order $i$ is started.
\end{itemize}

\subsection{Objective function}

\begin{equation*}
  \max \sum^n_{i = 1} \mathit{profit\_i} \: x_i
\end{equation*}

\subsection{Constraints}

Subject to
\begin{enumerate}
    \item Every processed order has finished baking before the maximum delivery time: 
    \begin{equation}
    \mathit{start\_{ij}}(j + \mathit{length\_i} - 1) \leq \mathit{max\_deliver\_i}, (1 \leq i \leq n, 1 \leq j \leq t)
    \end{equation}
    \item Every processed order has finished baking after the minimum delivery time:
    \begin{equation}
        j + \mathit{length\_i} - 1 \geq \mathit{start\_{ij}} \: \mathit{min\_deliver\_i}, (1 \leq i \leq n, 1 \leq j \leq t) 
    \end{equation}
    \item In every time slot, the space capacity is respected:
    \begin{equation}
        \sum^n_{i=1}\mathit{surface\_i} \: y_{ij} \leq \mathit{surface\_capacity}, (1 \leq j \leq t)
    \end{equation}
    % \item The auxiliary variable $\mathit{start}_i$ has the intended meaning:
    % \begin{equation}
    %     \mathit{start_i} \leq \mathit{max\_deliver}_i - \mathit{length}_i
    % \end{equation}
    \item We assign the correct amount of time slots or zero time slots to every order:
    \begin{equation}
        \sum^t_{j = 1} y_{ij} = x_i \: \mathit{length_i}, (1 \leq i \leq n)
    \end{equation}
    \item The time slots assigned to this order are contiguous (If an order $i$ starts in time slot $j$, it occupies the consecutive time slots from $j$ to $j + length_i$:
 %%   \begin{equation}
 %%       \mathit{start}_{ij} = 1 \doublearrow \sum^{j+\mathit{length_i}-1}_{k=j} y_{i,k} = \mathit{length}_i 
 %%   \end{equation}
        \begin{equation}
        \sum^{j+\mathit{length_i}-1}_{k=j} y_{i,k} \geq \mathit{start_{ij}} \mathit{length}_i, (1 \leq i \leq n, 1 \leq j \leq t - \mathit{length}_i + 1 )
    \end{equation}
    \item Each order only has one start point:
    \begin{equation}
    \sum^{t}_{j = 1} \mathit{start}_{ij} = x_i, (1 \leq i \leq n)
    \end{equation}
\end{enumerate}
\end{document}
